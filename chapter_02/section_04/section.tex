\documentclass[../../main]{subfiles}

\input{section_header.tex}

\begin{document}

\section{THU - JAN 29 2026} \label{sec:}

% \texttt{TODO: NEED TO UPDATE}

% \begin{itemize}
%     \item Boot process.
%     \item Architecture.
% \end{itemize}

\subsection{Architecture of Nucleo-H755ZI-Q}


\begin{center}
    \begin{figure}
        %text%
        \begin{verbatim}
                                   +-------------------------+
                                   |  +-------+   +-------+  |
                                   |  |  CM7  | + |  CM4  |  |
                 +--------+        |  +-------+   +-------+  |
                 | STLINK | <----> |                         |
                 +--------+        |                         |
                                   |        +-------------+  |
                                   |        | PERIPHERALS |  |
                                   |        +-------------+  |
                                   |                         |
                                   | NUCLEO-H755ZI-Q         |
                                   +-------------------------+
        \end{verbatim}
        \captionof{figure} {Rough architecture of Nucleo-H755ZI-Q.}
        \label{fig:stmNucleoRough}
    \end{figure}
\end{center}

Nucleo-H755ZI-Q has a asymmetric dual core architecture, having two cores
Arm Cortex M7 and Arm Cortex M4.

\alertNote{
    Asymmetric Multi-processing: One Master and One or More Slaves.
    Master does all the configuration and task scheduling.
}

\alertNote{
    Symmetric Multi-processing: All have equal access to memory and resources.
}

\begin{figure} [H]
    \centering
    \includegraphics [
        % height = 0.55\textheight,
        max width = \IGXMaxWidth,
        max height = \IGXMaxHeight,
        \IGXDefaultOptionalArgs,
    ] {pics/stm32_arch.pdf}
    \captionof{figure} {STM32H755ZI Architecture.}
    \label{fig:stm32Arch}
\end{figure}

\subsubsection{Sub-systems}

The entire MCU is divided into 3 major power domains.

\begin{itemize}
    \item D1: Where the M7 lies.
    \item D2: Where the M4 lies.
    \item D3: Where the RCC and PWR lies.
    \item BACKUP Domain: Where the EXTI lies.
\end{itemize}

%
% ### D1 Domain
%
% - CM7 (NVIC1)
% - Chrom-art Accelerator
% - SDMMC1
% - MDMA
% - DMA2D
% - QSPI
% - FMC
%
% - FLASH A
% - FLASH B
% - AXI SRAM
%
%
% ### D2 Domain
%
% - CM4 (NVIC2)
% - ETHERNET
% - SDMMC2
% - USB HS 1
% - USB HS 2
%
% - SRAM 1
% - SRAM 2
% - SRAM 3
%
% ### D3 Domain
%
% - RCC
% - PWR
% - BDMA
% - SRAM4
% - Backup SRAM
% - GPIOs
%

% ### Stop Mode
%
% <!-- # Something else -->
% <!---->
% <!-- This file has the FLASH "Drivers/CMSIS/Device/ST/STM32H7xx/Include/stm32h755xx.h". -->
%
% # Interrupts
%
% There are three types of signals are transferred between different blocks.
%
% - Wakeup Events
% - Signals
% - Interrupts
%
% ## Nested Vectored Interrupt Controllers (NVIC1 and NVIC2)
%
% ## EXTI

\alertImportant{
    From figure \ref{fig:stm32Arch}, we can see that the \texttt{EXTI}\footnote{External Interrupt Controller} is
    outside of the 3 domains. This enables to power down all 3 of these domains at the same time.
    Then \texttt{EXTI}'s domain can be kept active, and \texttt{EXTI} will be responsible for powering
    up the required domains to serve interrupts.
}

\subsubsection{D1 sub-system}

D1 domain is intended for high performance applications. Has a high bandwidth AXI bus matrix. The Cortex
M7 core is located here running at up to 480MHz with 16-Kbyte I-cache and 16-Kbyte D-cache.

\alertNote{
    \textbf{Internal Memory Resources:}

    \begin{itemize}
        \item AXI SRAM (512-Kbyte) mapped at address \texttt{0x2400 0000}.
        \item ITCM\footnote{Instruction Tightly Coupled Memory.} accessible through 64-bit ITCM bus.
        \item DTCM\footnote{Data Tightly Coupled Memory.} accessible through 2x32-bit DTCM bus.
        \item Flash memory: Two 1MB independent flash banks.
    \end{itemize}
}

\subsubsection{D2 sub-system}

The D2 domain is intended for generic peripheral usage, communication and data gathering that can be
processed later in D1 domain. It contains Cortex M4 core running at up to 240MHz.

\alertNote{
    \textbf{Internal Memory Resources:}

    \begin{itemize}
        \item AHB SRAM1 (128 KByte) mapped at address \texttt{0x3000 0000} and \texttt{0x1000 0000}.
        \item AHB SRAM2 (128 KByte) mapped at address \texttt{0x3002 0000} and \texttt{0x1002 0000}.
        \item AHB SRAM3 (32 KByte) mapped at address \texttt{0x3004 0000} and \texttt{0x1004 0000}.
    \end{itemize}
}

\subsubsection{D3 sub-system}

The D3 domain provides system management and low power operating modes feature. It contains peripherals
like PWR, RCC, ADCs, GPIOs. It has a AHB bus matrix, and a BDMA\footnote{Basic DMA.} It also contains
the communication peripherals like I2C, SPI, LPUART.

\alertNote{
    \textbf{Internal Memory Resources:}

    \begin{itemize}
        \item AHB SRAM4 (128 KByte) mapped at address \texttt{0x3000 0000}.
    \end{itemize}
}
\subsubsection{Peripherals}

From figure \ref{fig:stm32Arch} we can see that different peripherals are in different power domains.
Each domain has it's own bus matrix. There are inter domain buses for communication across domains. In
D1, 64-bit AXI\footnote{Advanced eXtensible Interface.} bus is used. D2 and D3 domains use 32-bit
AHB\footnote{Advanced High Performance} buses.

\alertNote{
    Most of the communication peripherals are located in the D2 domain. For example, USB, Ethernet,
    FDCAN, UART, SPI, and SD/MMC are located here. Most of the memories dedicated to I/O
    processing are also located here.
}

\end{document}
