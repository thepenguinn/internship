\documentclass[../../main]{subfiles}

\input{section_header.tex}

\begin{document}

\section{THU - JAN 29 2026} \label{sec:}

% \texttt{TODO: NEED TO UPDATE}

% \begin{itemize}
%     \item Boot process.
%     \item Architecture.
% \end{itemize}

\subsection{Architecture of Nucleo-H755ZI-Q}

%text%
\begin{minted}[breaklines, autogobble] {text}

                      +-------------------------+
                      |  +-------+   +-------+  |
                      |  |  CM7  | + |  CM4  |  |
    +--------+        |  +-------+   +-------+  |
    | STLINK | <----> |                         |
    +--------+        |                         |
                      |        +-------------+  |
                      |        | PERIPHERALS |  |
                      |        +-------------+  |
                      |                         |
                      | NUCLEO-H755ZI-Q         |
                      +-------------------------+
\end{minted}

Nucleo-H755ZI-Q has a asymmetric dual core architecture, having two cores
Arm Cortex M7 and Arm Cortex M4.

\alertNote{
    Asymmetric Multi-processing: One Master and One or More Slaves.
    Master does all the configuration and task scheduling.
}

\alertNote{
    Symmetric Multi-processing: All have equal access to memory and resources.
}

\subsubsection{Power Domains}

The entire MCU is divided into 3 major power domains.

\begin{itemize}
    \item D1: Where the M7 lies
    \item D2: Where the M4 lies
    \item D3: Where the RCC and PWR lies.
    \item BACKUP Domain: Where the EXTI lies.
\end{itemize}

%
% ### D1 Domain
%
% - CM7 (NVIC1)
% - Chrom-art Accelerator
% - SDMMC1
% - MDMA
% - DMA2D
% - QSPI
% - FMC
%
% - FLASH A
% - FLASH B
% - AXI SRAM
%
%
% ### D2 Domain
%
% - CM4 (NVIC2)
% - ETHERNET
% - SDMMC2
% - USB HS 1
% - USB HS 2
%
% - SRAM 1
% - SRAM 2
% - SRAM 3
%
% ### D3 Domain
%
% - RCC
% - PWR
% - BDMA
% - SRAM4
% - Backup SRAM
% - GPIOs
%

% ### Stop Mode
%
% <!-- # Something else -->
% <!---->
% <!-- This file has the FLASH "Drivers/CMSIS/Device/ST/STM32H7xx/Include/stm32h755xx.h". -->
%
% # Interrupts
%
% There are three types of signals are transferred between different blocks.
%
% - Wakeup Events
% - Signals
% - Interrupts
%
% ## Nested Vectored Interrupt Controllers (NVIC1 and NVIC2)
%
% ## EXTI

\alertImportant{
    From figure \ref{fig:stm32Arch}, we can see that the \texttt{EXTI}\footnote{External Interrupt Controller} is
    outside of the 3 domains. This enables to power down all 3 of these domains at the same time.
}

\begin{figure} [H]
    \centering
    \includegraphics [
        % height = 0.55\textheight,
        max width = \IGXMaxWidth,
        max height = \IGXMaxHeight,
        \IGXDefaultOptionalArgs,
    ] {pics/stm32_arch.pdf}
    \captionof{figure} {STM32 Architecture.}
    \label{fig:stm32Arch}
\end{figure}

\end{document}
