\documentclass[../../main]{subfiles}

\input{section_header.tex}

\begin{document}

\section{WED - FEB 04 2026} \label{sec:}

\subsection{UART Interfacing Exercise using STM32}

\paragraph{Exercise Question:} Configure three UART interfaces on the development board.
One UART is used as the ST-LINK virtual COM port and acts as the console interface.
The application prompts the user through the console to enter a numeric value between 100
and 10000 and to select a UART interface (UART2 or UART3).

Based on the selected UART:

\begin{itemize}
    \item If UART2 is selected, the console displays the data received from UART3 multiplied by 2.
    \item If UART3 is selected, the console displays the data received from UART2 multiplied by 2.
\end{itemize}

UART2 and UART3 are physically interconnected using jumper wires between their TX and RX
pins to enable bidirectional communication.

\subsubsection{Solution Overview}

The USART3 is connected to the ST-LINK's UART. We can use USART1 and USART2 as the other two
UARTs. All we need to do is to initialize these UARTs\footnote{Spoiler Alert: This can be done
through STM32CubeMX, and all we need to do is to start using them.}. Then we need to
set up the ISR\footnote{Interrupt Service Routine.} to process the received data. While the main
task will handle the user input.

\subsubsection{Solution Overview}

\begin{figure}
    \centering
    \includegraphics [
        max width = \IGXMaxWidth,
        max height = \IGXMaxHeight,
        \IGXDefaultOptionalArgs,
    ] {pics/stm_uart_block_diagram.pdf}
    \captionof{figure} {Connection block diagram.}
    \label{fig:uartBlockDiagram}
\end{figure}

% \texttt{TODO: BLOCK DIAGRAM}

\texttt{TODO: FLOW DIAGRAM}

% \texttt{TODO: UART EXERCISE}

\end{document}
