\documentclass[../../main]{subfiles}

\input{section_header.tex}

\begin{document}

\section{THU - FEB 05 2026} \label{sec:}

\texttt{TODO: FORENOON UART EXERCISE}

\subsection{USB On The Go}

STM32 Nucleo boards supports USB OTG in full speed mode via a USB \textbf{Micro-AB} connector (CN13).
The USB power switch (U18) is connected to $V_{BUS}$.

\alertNote{
    USB OTG (On The Go) means the electronics can act either as a host or a device.
    This is done via HNP\footnote{Host Negotiation Protocol}, and SRP\footnote{Session Request
    Protocol}. The board is also capable of ADP\footnote{Attach Detection Protocol} to
    determine the attachment of devices.
}

\alertWarning{
    Do not connect Micro-AB connector before properly powering the Nucleo-144 board. As the
    Micro-AB connector cannot power the board. There is a chance of current injection.
}

\alertImportant{
    STM32H755ZI is capable of USB HS\footnote{High Speed} but the Nucleo board only supports upto USB
    FS\footnote{Full Speed}. In order to use USB HS, we need to use an external through the ULPI\footnote{UTMI+
    Low Pin Interface} interface.
}

\subsubsection{USB 2.0: An Overview}

USB 2.0 uses 4 pins to enable communication. These pins are:

\begin{center}
    \begin{tabularx} {\textwidth} {
            >{\raggedright \arraybackslash}X
            *{4}{>{\centering \arraybackslash}X}
        }

        \toprule

        Pins & 5V & DM & DP & GND \\ \midrule

        Function & Power & \multicolumn{2}{c}{Differential Pairs} & Ground\\

        \bottomrule

    \end{tabularx}
\end{center}

The data is send as differential signal using DP and DM pins. This improves noise immunity.

\alertNote{
    USB 3.0 and above uses extra pins to enable higher speeds. Also note that the STM32H755ZI is only capable of
    USB 2.0.
}

\subsubsection{Different USB Speed Modes}

\begin{center}
    \begin{tabularx} {\textwidth} {
            >{\raggedright \arraybackslash}X
            *{3}{>{\centering \arraybackslash}X}
        }

        \toprule

        USB Modes & Low Speed (LS) & Full Speed (FS) & High Speed (HS) \\ \midrule

        Speed & 1.5 Mbit/s & 12 Mbit/s & 480 Mbit/s \\

        \bottomrule

    \end{tabularx}
\end{center}

\subsubsection{Different Modes}

USB features can be divided into three categories:

\begin{itemize}
    \item \textbf{General:}
    \item \textbf{Host-mode:} To be used as a host / master.
    \item \textbf{Device-mode:} To be used as a peripheral.
\end{itemize}

% \alertNote{
%
% }

\subsubsection{Endpoints}

% https://support.touch-base.com/Documentation/50434/USB-device-basics

\begin{itemize}
    \item \textbf{Control:}
        \begin{itemize} [label=]
            \item Commonly used for configuring the devices, retrieving data, sending commands and retrieving status. Usually
                in small size. And it is guaranteed to have reserved bandwidth.
        \end{itemize}
    \item \textbf{Interrupt:}
        \begin{itemize} [label=]
            \item For sending small amount of data at a fixed interval. Used for keyboards and mice. Have reserved bandwidth.
        \end{itemize}
    \item \textbf{Bulk:}
        \begin{itemize} [label=]
            \item For sending large amount of data. Can transfer huge amount of data without any data lose. Does not guaranteed
                to make it through in a specific amount of time. There won't be enough room on the bus to transfer huge amount of
                data, so the packet is split in multiple smaller packets.
        \end{itemize}
    \item \textbf{Isochronous:}
        \begin{itemize} [label=]
            \item For sending large amount of data. For devices that need continuous stream of data but can handle data loss. It
                is periodic.
        \end{itemize}
\end{itemize}

\alertNote{
    Interrupt and Isochronous are periodic and have reserved bandwidth. While Control and Bulk are asynchronous. Bulk does
    not have a reserved bandwidth.
}


% \subsubsection{Different Modes}





\end{document}
