\documentclass[../../main]{subfiles}

\input{section_header.tex}

\begin{document}

\section{FRI - FEB 06 2026} \label{sec:}

\subsection{USB On The Go (Continued)}

\subsubsection{Hardware Side Specific to STM32H755ZI}

The device is capable of USB HS through an external PHY. It already has an embedded USB FS PHY.
The nucleo board is capable of USB FS as it does not have an external USB HS PHY. In STM32H7, there
are two instances of OTG\_HS, namely OTG\_HS1 and OTG\_HS2. Only the OTG\_HS1 instance is capable of
achieving USB HS through an external PHY.

The device has 4Kb of USB data RAM with advanced FIFO\footnote{First in first out} control. This
memory can be partitioned by the user for each of the endpoints.

\subsubsection{OTG\_HS Core}

OTG\_HS receives the 48 MHz from the RCC\footnote{Reset and Clock Controller}. There are different
registers associated with the OTG\_HS core that helps to control the core.

\subsection{I2C: Inter-Integrated Circuit Interface of STM32H755ZI}

The I2C bus interface handles communication between the microcontroller and the serial I2C bus. The
interface provides multi-master capability, and controls all the I2C bus-specific sequencing, protocol,
arbitration and timing. It supports the following modes:

\begin{center}
    \begin{tabularx} {\textwidth} {
            >{\raggedright \arraybackslash}X
            *{3}{>{\centering \arraybackslash}X}
        }

        \toprule

        Modes & Standard Mode & Fast Mode & Fast Mode Plus \\ \midrule

        Speed & Up to 100 kHz & Up to 400 kHz & Up to 1 MHz \\

        \bottomrule

    \end{tabularx}
\end{center}

\alertNote{
    Note that the I2C bus interface is SMBus\footnote{System Management Bus} and
    PMBus\footnote{Power Management Bus} compatible. Also DMA can be used to reduce
    CPU overload.
}

\subsubsection{I2C: Hardware Overview}

\paragraph{Hardware Interface:} Hardware bus consists of two lines:

\begin{center}
    \begin{tabularx} {\textwidth} {
            >{\raggedright \arraybackslash}X
            *{2}{>{\centering \arraybackslash}X}
        }

        \toprule

        Pin & SDA & SCL \\ \midrule

        Function & Serial Data (Data Line) & Serial Clock (Clock Line) \\

        \bottomrule

    \end{tabularx}
\end{center}

\paragraph{Line Logic:}

These lines are open collector or open drain. Meaning the devices can only
drive them low. Default / Idle state is high.

\alertNote{
    We need to use a pull up resistor to pull these lines high. Usually
    we use 1 kOhm to 10 kOhm. Then the pull up current will be about 1mA
    or less.
}

\alertNote{
    Also the driving device need to sink at least 10mA or more current.
}

\paragraph{Pin Connection}

\begin{center}
    \begin{figure}
        %text%
        \begin{verbatim}
                  +------------+               +------------+
                  |            |               |            |
                  |            +<---> SDA <-+->|            |
                  |   MASTER   |            |  |   SLAVE    |
                  |            +--+-> SCL ---->|            |
                  |            |  |         |  |            |
                  +------------+  |         S  +------------+
                                  |         D
                                  |         A  +------------+
                                  |         |  |            |
                                  |         +->|            |
                                  |            |   SLAVE    |
                                  +-- SCL ---->|            |
                                               |            |
                                               +------------+
        \end{verbatim}
    \end{figure}
    \captionof{figure} {Typical connection diagram of SDA and SCL pins in a single master application.}
    \label{fig:}
\end{center}

\paragraph{SDA Line:}

This line is used to send the address from the master to slave. And to send and
receive data to and from the master and slaves.

\paragraph{SCL Line:}

This line is used to provide clock.

\paragraph{Voltage Levels:}

Supply Voltage: Ranging from 1.2V to 5.5V

\subsubsection{I2C: Communication Protocol}

Every data frame or the transaction is started with a \texttt{START} condition and ends with \texttt{STOP}
condition.

\paragraph{Clock Stretching:}

In I2C it is possible to stretch\footnote{Analogues to pausing the clock.} the SCL line. If any of the
device is slow to process the data, it can pull the SCL line low.

\paragraph{Arbitration:}

It's a mechanism to decide which master should take control of the I2C bus.

\texttt{TODO: FRAME}

% \subsubsection{}

\subsubsection{More about STM32H755ZI's I2C Interface: Implementation}

There are four I2C channels implemented in STM32H755ZI. They are: \texttt{I2C1},
\texttt{I2C2}, \texttt{I2C3}, and \texttt{I2C4}. All of these interfaces can
be used in the 3 modes. These interfaces can also generate interrupts. And they
can be enabled in the software.

\alertImportant{
    In order to use these interfaces in Fast-mode plus, we need to enable 20 mA output
    current in the control bit present in SYSCFG register.
}

\paragraph{Modes for these Interfaces:}

These interfaces can operate in one of the four modes:

\begin{center}
    \begin{tabularx} {\textwidth} {
            >{\raggedright \arraybackslash}m{4.5cm}
            *{4}{>{\centering \arraybackslash}X}
        }

        \toprule

        Modes & Slave Transmitter & Slave Receiver & Master Transmitter & Master Receiver \\

        \bottomrule

    \end{tabularx}
\end{center}

By default, the interface will operate in slave mode. Once it generates\footnote{From the software}
a \texttt{START} condition, it will automatically switches to master mode. If the arbitration losses
or a \texttt{STOP} condition is generated, it will automatically switches back to slave mode.

\subsubsection{I2C Initialization}

\begin{itemize}

    \item First of all we need to configure and enable I2C peripheral clock in
        the clock controller\footnote{In RCC}.

    \item Then the interface can be enabled by setting PE bit in I2C\_CR1 register.

\end{itemize}


\end{document}
