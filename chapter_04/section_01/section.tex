\documentclass[../../main]{subfiles}

\input{section_header.tex}

\begin{document}

\section{MON - FEB 09 2026} \label{sec:}

\subsection{UART - I2C Exercise}

\paragraph{Exercise Question:} Configure two UART interfaces, one I2C interfaces and an onboard
LED.

Based on the command received location:

\begin{itemize}
    \item If command received from ST-LINK's UART, then relay that to the other UART.
    \item If command received from the other UART, then relay that to the I2C.
    \item If command received from I2C, do as such to the LED.
\end{itemize}

Then connect the other UART and I2C to another STM32 and test.

\subsubsection{Solution Overview}

\begin{figure}
    \centering
    \includegraphics [
        max width = \IGXMaxWidth,
        max height = \IGXMaxHeight,
        \IGXDefaultOptionalArgs,
    ] {pics/stm_uart_i2c_led_block.pdf}
    \captionof{figure} {Connection block diagram of UART - I2C Exercise.}
    \label{fig:blockUartI2C}
\end{figure}

\subsubsection{Some Important Registers Related with I2C}

\texttt{TODO: MOVE THIS TO SOMEWHERE ELSE}

Registers:

\begin{itemize}
    \item I2C\_CR1

    \item I2C\_TIMINGR

        \begin{itemize}
            \item PRESC[3:0]
            \item SCLDEL[3:0]
            \item SDADEL[3:0]
        \end{itemize}

    \item I2C\_CR2
    \item I2C\_ISR

    \item I2C\_RXDR

    \item RXNE bit should be zero to receive.

    \item I2C\_TXDR

    \item TXE bit should be zero to send.

    \item NBYTES[7:0] in I2C\_CR2

    \item I2C\_OAR1
    \item I2C\_OAR2

\end{itemize}




\end{document}
