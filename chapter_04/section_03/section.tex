\documentclass[../../main]{subfiles}

\input{section_header.tex}

\begin{document}

\section{WED - FEB 11 2026} \label{sec:}

\subsection{SPI: Serial Peripheral Interface of STM32H755ZI}

Serial peripheral interface is used for synchronous serial communication in
embedded systems for short distance wired communication. The standard SPI supports
full duplex\footnote{Two way communication at the same time}. It follows a
master-slave architecture. The master orchestrates the communication to and from
one or more slave devices.

\subsubsection{Pin Description and Connection}

The standard SPI makes use of 4 pins in total. They are:

\begin{center}
    \begin{tabularx} {\textwidth} {
            >{\raggedright \arraybackslash}X
            *{4}{>{\centering \arraybackslash}X}
        }

        \toprule

        Pin & $\overline{\texttt{CS}}$ & \texttt{SCLK} & \texttt{MOSI}\footnote{Master Out Slave In.} & \texttt{MISO}\footnote{Master In Slave Out.} \\ \midrule

        Function & Chip Select & Serial Clock & Data to Slave from Master & Data to Master from Slave \\

        \bottomrule

    \end{tabularx}
\end{center}

\begin{center}
    \begin{figure}
        %text%
        \begin{verbatim}
             +-----------------+               +-----------------+
             |                 |               |                 |
             |            CS1# +-------------->| CS#             |
             |                 |               |                 |
             |            SCLK +--------+----->| SCLK            |
             |                 |        |      |                 |
             |            MOSI +-----+-------->| MOSI            |
             |                 |     |  |      |                 |
             |            MISO |<-+------------+ MISO            |
             |                 |  |  |  |      |                 |
             |  MASTER         |  |  |  |      |        SLAVE 1  |
             |            CS2# |  |  |  |      |                 |
             +-------------+---+  |  |  |      +-----------------+
                           |      |  |  |      +-----------------+
                           |      |  |  |      |                 |
                           |      |  |  +----->| SCLK            |
                           |      |  |         |                 |
                           |      |  +-------->| MOSI            |
                           |      |            |                 |
                           |      +------------+ MISO            |
                           |                   |                 |
                           +------------------>| CS#    SLAVE 2  |
                                               |                 |
                                               +-----------------+
        \end{verbatim}
        \captionof{figure} {Typical connection diagram of MOSI, MISO, SCLK, and $\overline{\mbox{CSx}}$ pins in a single master SPI application.}
        \label{fig:spiPinConnection}
    \end{figure}
\end{center}

From figure \ref{fig:spiPinConnection} we can see that the \texttt{SCLK}, \texttt{MOSI}, and \texttt{MISO} lines are common
to all the slaves. Each of the slave device will have it's corresponding $\overline{\texttt{CS}}$ line.

\alertNote{
    Native SPI does not support multi-master configuration. In order to use multi-master we need
    some way to perform master arbitration. STM32H755ZI's reference manual also does not mention
    about any inbuilt way to use the SPI in mutli-master mode.
}

\alertImportant{
    MISO line should be tri-stated, as it needs to be driven from different slaves at different
    times depending on the state of the corresponding chip select pin.
}

\alertNote{
    MISO and MOSI are also called SDI\footnote{Serial Data In.} and SDO\footnote{Serial Data Out.}
    respectively.
}

\subsubsection{Data Transmission}

The master initiates the transaction by asserting the corresponding $\overline{\texttt{CS}}$ line. The
clock is also provided by the master. Full duplex communication is carried out by the master and the
selected slave. The data is sampled or pushed out at the edge\footnote{Can be configured to rising or
falling edge, depending on the device.} of the \texttt{SCLK}.

\subsubsection{SPI Modes: Clock Polarity and Clock Phase}

SPI has four modes, the master can choose these modes using CPOL\footnote{Clock Polarity.} and
CPHA\footnote{Clock Phase} bits. See table \ref{tbl:spiModes} for the corresponding polarity and phase.

\begin{center}
    \begin{tabularx} {\textwidth} {
            >{\centering \arraybackslash}p{1cm}
            >{\centering \arraybackslash}p{1cm}
            >{\centering \arraybackslash}p{1cm}
            >{\centering \arraybackslash}X
            >{\centering \arraybackslash}X
            >{\centering \arraybackslash}X
            % *{4}{>{\centering \arraybackslash}X}
        }

        \toprule

        SPI Mode & CPOL & CPHA & Clock Polarity in Idle State & Clock Phase Used to Sample Data & Clock Phase Used to Shift the Data \\ \midrule

        0 & 0 & 0 & Logic Low  & Rising Edge  & Falling Edge \\
        1 & 0 & 1 & Logic Low  & Falling Edge & Rising Edge  \\
        2 & 1 & 0 & Logic High & Falling Edge & Rising Edge  \\
        3 & 1 & 1 & Logic High & Rising Edge  & Falling Edge \\

        \bottomrule

    \end{tabularx}
    \captionof{table}{CPOL and CPOH states and corresponding clock polarity and edge selections for sampling and shifting data.}
    \label{tbl:spiModes}
\end{center}



\end{document}
