\documentclass[../../main]{subfiles}

\input{section_header.tex}

\begin{document}

\section{THU - FEB 13 2026} \label{sec:}

\subsection{Hardware Design Exercise}

\subsubsection{Requirements}

\begin{enumerate}

    \item Product Overview.
    \item Block Diagram.
    \item Power Calculation.
    \item Interfaces.
    \item Interfacing Diagram.
    \item PCB layout Guidelines.
    \item Mechanical Guidelines.

\end{enumerate}

\subsubsection{Design Problem:}

Design a system to control a camera with STM32. The system should take input from a joystick and
6 push buttons. For debugging purposes we need an OLED display and JTAG interface.

% \begin{enumerate}
%
% \end{enumerate}

\subsubsection{STM32H745ZI: Power}

The required supply voltage is 1.2V to 3.3V.

\subsubsection{STM32H745ZI: Interfaces}

\begin{center}
    \begin{tabularx} {\textwidth} {
            >{\raggedright \arraybackslash}X
            *{3}{>{\raggedright \arraybackslash}X}
        }

        \toprule

        Device & \multicolumn{3}{l}{Available Interfaces} \\ \midrule

        STM32H745ZI & 4 x I2C & 4 x USART & 6 x SPI \\
        & 2 x CAN & 2 x USB OTG & Ethernet \\
        & HDMI & 2 x SD/SDIO/MMC & MDIO Slave \\

        \bottomrule

    \end{tabularx}
    \captionof{table}{Some of the interfaces available in STM32H745ZI.}
\end{center}

\subsubsection{Joy Stick 67A-DF-IC-060C: Power}

Required power supply is $3.3\si{V} \pm 0.3\si{V}$.

\subsubsection{OLED: Power}

Required power supply is $3.3\si{V}$ to $5\si{V}$. We will be driving it using $3.3\si{V}$.

\subsubsection{Interfaces Used in the System}

\begin{center}
    \begin{tabularx} {\textwidth} {
            >{\raggedright \arraybackslash}p{6cm}
            *{3}{>{\raggedright \arraybackslash}X}
        }

        \toprule

        Device & \multicolumn{3}{l}{Interfaces Used} \\ \midrule

        STM32H745ZI & I2C & UART & JTAG \\
        Joy Stick 67A-DF-IC-060C & I2C & & \\
        OLED Display & I2C & & \\
        External Camera & UART & & \\
        External Debugger & JTAG & & \\

        \bottomrule

    \end{tabularx}
    \captionof{table}{Devices and Interfaces Used.}
\end{center}

% \subsubsection{Switches: Functions}
%
% \begin{enumerate}
%     \item HOME
%     \item RESET
%     \item LEFT
%     \item RIGHT
%     \item UP
%     \item DOWN
% \end{enumerate}

% \subsubsection{Design: Parts / Components}
%
% \begin{enumerate}
%     \item 1.3 inch OLED: 3647-26295-ND
% \end{enumerate}

\subsubsection{Block Diagram of the Entire System}

\begin{figure}
    \centering
    \includegraphics [
        max width = \IGXMaxWidth,
        max height = \IGXMaxHeight,
        \IGXDefaultOptionalArgs,
    ] {pics/stm_joy_cam_block_diagram.pdf}
    \captionof{figure} {Block Diagram.}
    \label{fig:}
\end{figure}

\subsubsection{Power Calculation}

\begin{figure}
    \centering
    \includegraphics [
        width = 0.98\textwidth,
        max width = \IGXMaxWidth,
        max height = \IGXMaxHeight,
        \IGXDefaultOptionalArgs,
    ] {pics/stm_cam_power_estimate.pdf}
    \captionof{table} {Power estimate of the system.}
    \label{tbl:powerEstimate}
\end{figure}

The entire system consumes atmost about $1680\si{mW}$ of power. Refer table \ref{tbl:powerEstimate},
for the break down of supply rails and the current consumption of each of the devices from the
respective rails.

\alertImportant{
    In table \ref{tbl:powerEstimate}, power dissipation of buck converter LM2596 is not
    added.
}

\subsubsection{Interfacing Diagram: Joy Stick}

\begin{figure}
    \centering
    \includegraphics [
        width = 0.4\textwidth,
        max width = \IGXMaxWidth,
        max height = \IGXMaxHeight,
        \IGXDefaultOptionalArgs,
    ] {pics/stm_interface_joy.png}
    \captionof{figure} {Interfacing diagram of joy stick with STM32H745ZI.}
    \label{fig:interJoy}
\end{figure}

\alertImportant{
    Pull-ups for \texttt{SDA} and \texttt{SCL} are taken as $1.5\si{k\ohm}$ as recommended in
    the interfacing diagram from the datasheet. The pull-up for \texttt{INTn} is taken as
    $10\si{k\ohm}$. See figure \ref{fig:joyInterGuide} for more info.
}

As we can see from the figure \ref{fig:interJoy} the joy stick is interfaced
through I2C, with an extra interrupt pin.

\alertNote{

    Along with \texttt{SCL} and \texttt{SDA} pins, the joy stick also provides an \texttt{INTn}
    pin. Which can be used to generate an interrupt from the device. Also note that these
    pins\footnote{\texttt{SCL}, \texttt{SDA} and \texttt{INTn}.} need to be pulled high. Along
    with this, there is also an \texttt{A1n} pin to select the I2C addresses. That can be grounded
    to choose the address \texttt{0x82}. Please refer to the diagram \ref{fig:joyInterGuide} for more info.

    \begin{figure} [H]
        \centering
        \includegraphics [
            width = 0.7\textwidth,
            max width = \IGXMaxWidth,
            max height = \IGXMaxHeight,
            \IGXDefaultOptionalArgs,
        ] {pics/joy_interface_guideline.png}
        \captionof{figure} {Interfacing guidelines given in the datasheet of Joy Stick.}
        \label{fig:joyInterGuide}
    \end{figure}
}

\subsubsection{Interfacing Diagram: OLED}

\begin{figure}
    \centering
    \includegraphics [
        width = 0.4\textwidth,
        max width = \IGXMaxWidth,
        max height = \IGXMaxHeight,
        \IGXDefaultOptionalArgs,
    ] {pics/stm_interface_oled.png}
    \captionof{figure} {Interfacing diagram of oled display with STM32H745ZI.}
    \label{fig:interOled}
\end{figure}

\alertImportant{
    We will be using the same I2C bus used by the Joy stick module. Therefore,
    as mentioned there, the $1.5\si{k\ohm}$ is taken as the pull-ups for \texttt{SDA} and \texttt{SCL}
    lines.
}

As we can see from the figure \ref{fig:interOled} the OLED display is interfaced
through I2C.

\alertNote{
    The I2C address of OLED module is \texttt{0x3C}, so there will be no clash with
    joy stick module.
}

\subsubsection{Interfacing Diagram: Camera}

\begin{figure}
    \centering
    \includegraphics [
        width = 0.4\textwidth,
        max width = \IGXMaxWidth,
        max height = \IGXMaxHeight,
        \IGXDefaultOptionalArgs,
    ] {pics/stm_interface_camera.png}
    \captionof{figure} {Interfacing diagram of camera with STM32H745ZI.}
    \label{fig:interCamera}
\end{figure}

As we can see from the figure \ref{fig:interCamera} the camera is interfaced
through UART.

\subsubsection{Interfacing Diagram: JTAG}

\begin{figure}
    \centering
    \includegraphics [
        width = 0.5\textwidth,
        max width = \IGXMaxWidth,
        max height = \IGXMaxHeight,
        \IGXDefaultOptionalArgs,
    ] {pics/stm_interface_jtag.png}
    \captionof{figure} {Interfacing diagram of JTAG with STM32H745ZI.}
    \label{fig:interJtag}
\end{figure}

\alertImportant{
    In figure \ref{fig:interJtag} the pull-ups to \texttt{TCK} and \texttt{nRESET} are $47\si{k\ohm}$.
    And the rest of the pull-ups are $100\si{k\ohm}$.
}

\alertNote{
    In figure \ref{fig:interJtag}, you can add a series $33\si{\ohm}$ resistor on \texttt{TDO} line,
    close to the processor for impedance matching as mentioned in the specification of JTAG.
}

\subsubsection{PCB Layout Guidelines: LM2596}

\begin{figure} [H]
    \centering
    \includegraphics [
        width = 0.8\textwidth,
        max width = \IGXMaxWidth,
        max height = \IGXMaxHeight,
        \IGXDefaultOptionalArgs,
    ] {pics/lm2596_layout_guideline.png}
    \captionof{figure} {PCB layout guideline for LM2596 in fixed output mode.}
    \label{fig:lm2596Layout}
\end{figure}

\subsubsection{PCB Layout Guidelines: JTAG Connector}

It is recommended to place the JTAG connector no more than 5 inches from the
processor.

% joy_interface_guideline.png
% lm2596_layout_guideline.png
% stm_cam_power_estimate.pdf
% stm_interface_camera.png
% stm_interface_joy.png
% stm_interface_jtag.png
% stm_interface_oled.png
% stm_joy_cam_block_diagram.pdf

% \begin{figure}
%     \centering
%     \includegraphics [
%         max width = \IGXMaxWidth,
%         max height = \IGXMaxHeight,
%         \IGXDefaultOptionalArgs,
%     ] {tikzpics/endStmJoyStickInterface.pdf}
%     \captionof{figure} {Interfacing Diagram of STM32H745ZI and Joy Stick.}
%     \label{fig:}
% \end{figure}

% h745

\end{document}
