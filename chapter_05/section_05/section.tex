\documentclass[../../main]{subfiles}

\input{section_header.tex}

\begin{document}

\section{MON - FEB 20 2026} \label{sec:}

\subsection{Introduction to FreeRTOS}

For most of the embedded systems, timing is really important. It means the system needs
to respond to the events as fast as possible. There are two types of embedded systems
based on the timing requirement: Hard Real Time and Soft Real Time.

\alertNote{
    \begin{itemize} [label=]
        \item \textbf{Hard Real Time:} Timely response should be strictly enforced, otherwise
            something catastrophic will happen.
        \item \textbf{Soft Real Time:} It's fine to miss the timely response to an event. It would
            only cause a minor inconvenience, nothing catastrophic will happen.
    \end{itemize}
}

It is important to design the software that driving the system with this timing requirement in mind.
Once the application software reaches a certain level of complexity it will be hard to manage this
timing requirements.

Another important thing that all embedded systems should adhere is determinism. The system should
be deterministic in all the scenarios. Otherwise it will be hard to make sure the system will meet
the timing requirements.

One way to solve this issue\footnote{without loosing sanity} is to decouple the application software
from the timing requirements. Then manage the timing externally. Thus born the scheduler.

\subsubsection{Schedulers}



\end{document}
