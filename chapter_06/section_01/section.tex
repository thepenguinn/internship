\documentclass[../../main]{subfiles}

\input{section_header.tex}

\begin{document}

\section{MON - FEB 23 2026} \label{sec:}

\subsection{IPC: Inter Process Communication}

IPC encompasses different mechanisms used to provide communication between different processes,
that may be running on the same core or different cores. Inorder to synchronize different processes
there needs to be some mechanism at the software level and at the hardware level\footnote{If the
system is a multi core system.}.

\subsection{AMP: Asymmetric Multi-Processing}

Asymmetric Multi-Processing is a multi processing structure wherein one processor is called the
master and controls the other subordinate processors\footnote{They are also called slave processors.}.
In this configuration, all the OS activities and task scheduling are done by the master processor
of the system.

\alertNote{
    \textbf{Some of the advantages\footnote{Ofcourse, there are also disadvantages picking AMP.} of AMP:}

    \begin{itemize}
        \item Simplified Control.
        \item Reduced Resource Conflicts.
        \item Specialized Processors.
    \end{itemize}
}


\subsection{Semaphores}

\subsection{Hardware Semaphores}

\subsection{MPU: Memory Protection Unit}

\end{document}
